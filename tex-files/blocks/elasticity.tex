\section{Resource Management}
\label{sec:resource}

\noindent \textbf{\\Purpose:} 

Failures are the norm. there is a need for a component to detect and recover from failures. since stream jobs are every running jobs, this is an essential component. different modes, provide different guarantees.

\subsection{Failure Detection}
\Fix{maybe talk about heartbeat, pingpong, or piggy back mechanisms?}
\subsection{Failure Tolerance}
\Fix{talk about mechanism used for availability. Maybe the independent design structure that one some of the paths are unavailable not the whole dataflow.}
\subsection{Failure Recovery}

\noindent \textbf{\\Properties:}


\begin{figure}[h]
	\centering
	\includegraphics[width=0.45\linewidth]{guarantees.jpg}
	\caption{Spectrum of various failure properties and the trade-off among them.}
	\label{fig:failure}
\end{figure}

\begin{itemize}
	\item precise recovery:  \borrowed{hides the effects of a failure perfectly, except
		for some transient increase in processing latency, and is
		well-suited for applications that require the post-failure output
		be identical to the output without failure. Many financial
		services applications have such strict correctness requirements.}
	\item rollback recovery: \borrowed{ avoids information loss without guaranteeing
		precise recovery. The output produced after a failure
		is “equivalent” to, but not necessarily the same as, the
		output of an execution without failure. The output may also
		contain duplicate tuples. To avoid information loss, the system
		must preserve all the necessary input data for the backup
		server to rebuild (from its current state) the primary’s state at
		the moment of failure. Rollback recovery is thus appropriate
		for applications that cannot tolerate information loss but
		may tolerate imprecise output caused by the backup server
		reprocessing the input somewhat differently than the primary
		did. Example applications include those that monitor specific
		conditions (e.g., fire alarms, theft prevention through asset
		tracking). We show in Section 6 that this recovery guarantee
		can be provided more efficiently than precise recovery both
		in terms of runtime overhead and recovery speed.}
	\item gap recovery: \borrowed{ our weakest recovery guarantee, addresses
		the needs of applications that operate solely on the most recent
		information (e.g., sensor-based environment monitoring),
		where dropping some old data is tolerable for reduced
		recovery time and runtime overhead}
\end{itemize}


\noindent \textbf{\\Solutions \& trade-offs:}

\borrowed{ employs a different combination of redundant
	computation, checkpointing, and remote logging, they offer
	different tradeoffs between runtime overhead and recovery
	performance.
}
\begin{itemize}
	\item \borrowed{all following items borrowed!}
	\item amnesia, a lightweight scheme that provides
	\textbf{gap} recovery without any runtime overhead (Section 4).
	\item  passive standby and active standby, two
	process-pairs [4, 10] approaches tailored to stream processing.
	In passive standby, each primary server (a.k.a. node) periodically
	reflects its state updates to its secondary node. In
	active standby, the secondary nodes process all tuples in parallel
	with their primaries. 
	\item  propose upstream backup,
	an approach that significantly reduces runtime overhead compared
	to the standby approaches while trading off a small
	fraction of recovery speed. 
	
	Last two items can be both roll-back and precise recovery.
	
	\Fix{are there other approaches?}
\end{itemize}


\begin{table}[h]
	\tbl{Solutions to \Fix{XXX} and their trade-offs	\label{table:failure-summary}}{
		\tiny		
		\begin{tabular}{p{.15 \linewidth}|p{.3 \linewidth}|p{.3 \linewidth}|p{.1 \linewidth}}%
			
			\hline
			\rowcolor[HTML]{E0E0E0} 
			Solution	&Pros&Cons & Usecase
			\csvreader[head to column names]{tables/template-summary.csv}{}% use head of csv as column names
			{\\\hline\textbf{\csvcoli} & \csvcolii & \csvcoliii & \csvcoliv}% specify your coloumns here
			\\ \hline
		\end{tabular}	
	}
\end{table}


\noindent \textbf{\\Future Direction:}  