\section{Communication Mechanism}
\label{sec:communication}


\noindent\textbf{Definition and Purpose:} To enable large scale processing, operators are distributed across multiple machines. This distribution necessitates a \textit{communication mechanism} connecting operators to each other, and enabling data items to flow between operators. This section covers the desired properties from the communication mechanism and the current technologies used.



\noindent \textbf{\\Properties:}
%Various communicatStream processing frameworks utlize diverse comminucation mechanism based on properties they require from the communication
%
Among the various communication mechanism \Fix{cite}, the main differentiator is the properties and guarantees they provide. 
Based on these properties, stream processing frameworks employ the mechanism that best fits their requirements. The main properties of a communication mechanism are:

\begin{itemize}
	
	
	\item \textbf{Push/Pull} Which direction initiates communication? The sender can push the items downstream, the receiver can pull from upstream, or a mixed push-pull approach can be used, such as a publish-subscribe mechanism. \Fix{cite}
	
	
	\item \textbf{Fault-tolerance:} Can the mechanism handle failures? In presence of a failure, is availability impacted or can data get lost? \Fix{is this a valid point? should it be here?} \fkc{Should we move this to accuracy/processing semantics?}
	\item \textbf{Replayability}: Can the message be replayed? If so, for how far in the past and how large is the buffer?
	%\item \textbf{buffering: } are there any buffering capabilities. How large is the buffer?
	
	\item \textbf{Ordering:} Are there any ordering guarantees, such as total ordering,  first in first out (FIFO) or 
	
	\item \textbf{ direct or proxy}: is the communication direct from one machine to another or through a second entity?
	\item \textbf{ batch or online: } is the communication done in batches or sent as soon as available?
\end{itemize}

\noindent \textbf{\\Solutions, trade-offs and usecases:}

\begin{itemize}
	\item RPC methods
	\item publish-subscribe through 3rd entity (kafka)
	\item direct pub-sub. -- zeroMQ
	\Fix{TODO: study this in more detail}
	
\end{itemize}



\noindent \textbf{\\Future Direction:}  
.




