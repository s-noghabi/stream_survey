\section{Introduction}
\label{sec:intro}

In the era of big data, humongous amounts of data is produced every second from diverse sources such as social networks, sensors, IoT, etc. In the majority of use-cases, processing this massive amount of data  is latency sensitive (sub-seconds to minutes), either because of  cost benefits (e.g., every second latency in Google cost \Fix{X}) or the need for a critical action (e.g., detecting attacks and reacting to it).

To address this need, many stream processing systems have been developed with the ability to process huge amounts of data on a distributed cluster of machines in  near real-time fashion \Fix{cite}.  


\textbf{Contributions of this paper: }

\begin{itemize}
	\item Outline the essential building blocks (BB) in stream processing
	\item \textbf{for each BB: }
	\begin{itemize}
		\item \textbf{Definition: }  define the BB (\textit{What is it?})
		\item \textbf{Purpose: } describe the purpose, importance of it (\textit{What does it solve?})
		\item \textbf{Properties:} what are the different properties/modes for this component (\textit{What variations does it have?})
		\item \textbf{Solutions :}
		\begin{itemize}
		\item \textbf{approaches} survey the current  and past techniques for that BB (\textit{How to do it?}). 
		\item \textbf{trade-off} show the trade-off, pro and con of  each technique (\textit{What is the good/bad of each solution?}) 
		\item \textbf{Usecase: } describe which system uses this approach BB (\textit{Who uses it?})
		\end{itemize}
		\item \textbf{Future Direction:} discuss the remaining challenges and potential improvement (\textit{What is missing?})
	\end{itemize}
	\item the interaction among building blocks.
	\item describe the top-5 most popular current stream processing solutions and the components in each.

\end{itemize}






